% Required *.tex Files:
%   - dependencies.tex

% Required Packages:
%   - {amsmath, amsthm, amssymb, amsfonts}
%   - {xifthen, xparse, xargs, xintexpr, xfrac, xstring}
%   - {dashbox, adjustbox, fancybox}
%   - {mathtools, etoolbox}
%   - {gensymb, stmaryrd, relsize, bm, dsfont}
%   - {array}
%   - {pgffor}
%   - {listings}

% Notes:
%   -  `\mathrel`  :  relation
%   -  `\mathbin`  :  binary operation
%   -  `\mathop`   :  unary operation
%   -  `\mathord`  :  var
%   -   always wrap a call to `\mathlarger` or `\mathsmaller` with one of the above
%
%   -  `\lbrack` := `\left[`       `\rbrack` := `\right]`
%   -  `\lparen` := `\left(`       `\rparen` := `\right)`
%   -  `\lbrace` := `\{`           `\rbrace` := `\}`
%   -    ...     :=  ...              ...    :=  ...
%
%   -  `\bm` preserves spacing
%   -  `\mathlarger`  and  `\mathsmaller`  do *not* preserve spacing
%   -  `\bm{\left<}`  works,    `\left\bm{<}`  does not work
%
%   -  `\implies`    :=  `==>`
%   -  `\impliedby`  :=  `<==`
%
%   -  `\DeclarePairedDelimiter{cmd}{leftdelim}{rightdelim}`
%   -  `\DeclarePairedDelimiterX{cmd}[args]{leftdelim}{rightdelim}{body}`
%   -  `\DeclarePairedDelimiterXPP{cmd}[args]{precode}{leftdelim}{rightdelim}{postcode}{body}`
%
%   -  `\bm{ \sqrt{\mbox{\unboldmath$xyz$}} }`
%   -  `\bm{ \unboldmath{xyz}} }`
%   -  `\bm{\sqrtsign}{xyz}`
%
%   -  `\bm{\mathord{\mathlarger{\left(\right.}}} #1  \bm{\mathord{\mathlarger{\left.\right)}}}`


%%%%%%%%%%%%%%%%%%%%%%%%%%%%%%%%%%%%%%%%%%%%%%%%%%%%%%%%%%%%%%%%%%%%%%%%%%%%%%%
%%%%                       math operators & commands                       %%%%
%%%%%%%%%%%%%%%%%%%%%%%%%%%%%%%%%%%%%%%%%%%%%%%%%%%%%%%%%%%%%%%%%%%%%%%%%%%%%%%

%
%  General
%
\newcommand{\Part}[1]{\vspace{.10in}{\bf (#1)}}
\newcommand{\Justif}[2]{&{#1}&\text{#2}}
\newcommand{\Justifyy}[2]{&{#1}&#2}
\newcommand{\IndentItem}{\setlength\itemindent{23mm}\setlength\hangindent{26mm}}
\newcommand{\Question}[2]{\vspace{.25in} \hrule\vspace{0.5em}
\noindent{\normalfont\large\bfseries#1: #2} \vspace{0.5em}
\hrule \vspace{.10in}}
\newcommand{\Solution}{\textbf{\normalsize\ Solution}}


%
%  Basics
%
\DeclarePairedDelimiter{\parens}{\lparen}{\rparen}
\DeclarePairedDelimiter{\braces}{\lbrace}{\rbrace}
\DeclarePairedDelimiter{\bracks}{\lbrack}{\rbrack}
\DeclarePairedDelimiter{\verts}{\lvert}{\rvert}
\DeclarePairedDelimiter{\dverts}{\lVert}{\rVert}
\DeclarePairedDelimiter{\angles}{\langle}{\rangle}
\DeclarePairedDelimiter{\ceils}{\lceil}{\rceil}
\DeclarePairedDelimiter{\floors}{\lfloor}{\rfloor}
\DeclarePairedDelimiterX{\tuples}[2]{\lparen}{\rparen}{#1,\,#2}
\newcommand{\abs}[1]{\verts*{#1}}

\DeclarePairedDelimiterXPP{\Parens}[1]{\bm\begingroup}{\mathlarger{(}}{\mathlarger{)}}{\endgroup}{\mbox{\unboldmath{$#1$}}}
\DeclarePairedDelimiterXPP{\Braces}[1]{\bm\begingroup}{\mathlarger{\lbrace}}{\mathlarger{\rbrace}}{\endgroup}{\mbox{\unboldmath{$#1$}}}
\DeclarePairedDelimiterXPP{\Bracks}[1]{\bm\begingroup}{\mathlarger{[}}{\mathlarger{]}}{\endgroup}{\mbox{\unboldmath{$#1$}}}
\DeclarePairedDelimiterXPP{\Verts}[1]{\bm\begingroup}{\mathlarger{\lvert}}{\mathlarger{\rvert}}{\endgroup}{\mbox{\unboldmath{$#1$}}}
\DeclarePairedDelimiterXPP{\Dverts}[1]{\bm\begingroup}{\mathlarger{\lVert}}{\mathlarger{\rVert}}{\endgroup}{\mbox{\unboldmath{$#1$}}}
\DeclarePairedDelimiterXPP{\Angles}[1]{\bm\begingroup}{\mathlarger{\langle}}{\mathlarger{\rangle}}{\endgroup}{\mbox{\unboldmath{$#1$}}}
\DeclarePairedDelimiterXPP{\Ceils}[1]{\bm\begingroup}{\mathlarger{\lceil}}{\mathlarger{\rceil}}{\endgroup}{\mbox{\unboldmath{$#1$}}}
\DeclarePairedDelimiterXPP{\Floors}[1]{\bm\begingroup}{\mathlarger{\lfloor}}{\mathlarger{\rfloor}}{\endgroup}{\mbox{\unboldmath{$#1$}}}
\DeclarePairedDelimiterXPP{\Tuples}[2]{\bm\begingroup}{\mathlarger{(}}{\mathlarger{)}}{\endgroup}{\mbox{\unboldmath{$#1,\,#2$}}}
\newcommand{\Abs}[1]{\Verts*{#1}}


%
%  Logic
%
\newcommand{\suchthat}{\mathrel{\mathsf{st}}}
\newcommand{\then}{\bm{\mathop{\mathlarger{,}}}\,}
\newcommand{\given}{\bm{\mathord{\mathlarger{\left. \middle\vert \right.}}}}
\newcommand{\by}{\bm{\mathord{\mathlarger{;}}}\ }


%
%  Sets
%
%\DeclareMathOperator*{\setunion}{\cup} % TODO
%\DeclareMathOperator*{\setintersection}{\cap} % TODO
%\DeclareMathOperator*{\setdifference}{\mathlarger{\setminus}} % TODO
\DeclareMathOperator{\powerset}{\mathcal{P}}

% TODO Set TODO
\newcommand{\Set}[1]{\bm{\mathord{\mathlarger{\mleft\lbrace\vphantom{#1}\mright.}}} #1 \bm{\mathord{\mathlarger{\mleft.\vphantom{#1}\mright\rbrace}}}}

%\newcommandx{\SetUnion}[2]{} % TODO
%\newcommandx{\SetIntersection}[2]{} % TODO
%\newcommandx{\SetDifference}[2]{} % TODO
\newcommand{\PowerSet}[1]{\powerset\parens*{#1}}
\newcommand{\Cardinality}[1]{\verts*{#1}}


%
%  Functions
%
\DeclareMathOperator{\inverse}{^{-1}}
\newcommand{\Inverse}[1]{{#1}\inverse}

\newcommandx{\Func}[2][1=f]{\operatorname{#1} \parens*{#2}}
\newcommandx{\FuncComp}[3][1=f, 2=g]{\parens*{#2 \circ #1} \parens*{#3}}

\newcommandx{\Inj}[1][1=r]{% % TODO (find better symbol)
  \begin{switch}{#1}
    \caseof{l}{\hookleftarrow}
    \caseof{r}{\hookrightarrow}
    \caseof{b}{\hookleftrightarrow}
  \end{switch}
}
\newcommandx{\Surj}[1][1=r]{% % TODO (find better symbol)
  \begin{switch}{#1}
    \caseof{l}{\leftarrowtriangle} % {req: stmaryrd}
    \caseof{r}{\rightarrowtriangle} % {req: stmaryrd}
    \caseof{b}{\leftrightarrowtriangle} % {req: stmaryrd}
  \end{switch}
}
\newcommand{\Bij}{\leftrightarrow} % TODO (find better symbol)


%
%  Combinatorics
%
\newcommand{\perm}[1]{#1!}
\newcommand{\kperm}[3]{% % TODO (change to \newcommandx{}... )
  \begin{switch}{#1}
    \caseof{rep}{\frac{#2!}{#2^{#3}\ \parens*{#2-#3}!}}
    \caseof{norep}{\frac{#2!}{\parens*{#2-#3}!}}
    \caseof{}{#2^{#3}}
  \end{switch}
}
% \newcommand{\kperm}[2]{\frac{#1!}{(#1-#2)!}}
\newcommand{\kselect}[2]{\frac{#1!}{#2!\ \parens*{#1-#2}!}}
\renewcommand{\choose}[2]{\binom{#1}{#2}}


%
%  Algorithms
%
\newcommand{\Alg}[1]{\textsc{\bfseries \footnotesize #1}}


%
%  Asymptotic Analysis
%
\DeclareMathOperator{\bigo}{\mathcal{O}}
\DeclareMathOperator{\bigomega}{\Omega}
\DeclareMathOperator{\bigtheta}{\Theta}

\newcommand{\BigO}[1]{\bigo \parens*{#1}}
\newcommand{\BigOmega}[1]{\bigomega \parens*{#1}}
\newcommand{\BigTheta}[1]{\bigtheta \parens*{#1}}

\newcommandx{\AsymptoticBound}[2][1=upper]{%
  \begin{switch}{#1}% bound: {upper, lower, both}
    \caseof{upper}{\BigO{#2}}
    \caseof{lower}{\BigOmega{#2}}
    \caseof{both}{\BigTheta{#2}}
  \end{switch}
}


%
%  Linear Algebra
%
\DeclareMathOperator{\spans}{\mathrm{span}}
\DeclareMathOperator{\nullity}{\mathrm{nullity}}
\DeclareMathOperator{\rank}{\mathrm{rank}}
\DeclareMathOperator{\nullspace}{\mathrm{nullspace}}
\DeclareMathOperator{\colspace}{\mathrm{colspace}}
\DeclareMathOperator{\rowspace}{\mathrm{rowspace}}
\DeclareMathOperator{\range}{\mathrm{range}}
\DeclareMathOperator{\trace}{\mathrm{Tr}}
\DeclareMathOperator{\identity}{\mathbb{I}}
\DeclarePairedDelimiterX{\norm}[1]{\lVert}{\rVert}{\ifblank{#1}{\:\cdot\:}{#1}}
\DeclarePairedDelimiterXPP{\pnorm}[2]{}{\lVert}{\rVert}{\ifblank{#2}{_{L^{2}}}{_{L^{#2}}}}{\ifblank{#1}{\:\cdot\:}{#1}}
\DeclarePairedDelimiterX{\innerprod}[2]{\langle}{\rangle}{#1,\,#2}

\let\arrowvec\vec
\renewcommand{\vec}[1]{\bm{#1}}

\newcommand{\Scalar}[1]{#1}
\newcommand{\Vector}[1]{\vec{#1}}
\newcommand{\Matrix}[1]{\bm{#1}}
\newcommand{\T}[0]{^{T}\!}
\newcommand{\Transpose}[1]{\matrix{#1}\T}
\newcommand{\CT}[0]{^{\dag}}
\newcommand{\ConjugateTranpose}[1]{\matrix{#1}\CT}


%
%  Calculus
%
\DeclareMathOperator{\dx}{\mathrm{d}x}
%\DeclareMathOperator*{}{}

\newcommand{\Deriv}[1]{\frac{\mathrm{d}}{\mathrm{d}x} (#1)}
\newcommand{\PDeriv}[2]{\frac{\partial}{\partial\ #1} (#2)}


%
%  Physics
%
\newcommand{\uvec}[1]{\bm{\hat{\textbf{#1}}}}
\newcommand{\ivec}{\uvec{\i}}
\newcommand{\jvec}{\uvec{\j}}
\newcommand{\kvec}{\uvec{k}}


%
%  Probability Theory
%
\newcommand{\indep}{\bm{\mathbin{\mathlarger{\perp}}}}
\newcommand{\condindep}{\bm{\mathbin{\mathlarger{\dperp}}}}

\DeclareMathOperator{\pdf}{f} % TODO
\DeclareMathOperator{\pmf}{p} % TODO
\DeclareMathOperator{\cdf}{F} % TODO

\DeclareMathOperator{\probability}{\mathbf{P}}
\DeclareMathOperator{\expectation}{\mathbb{E}}
\DeclareMathOperator{\variance}{\mathrm{Var}}
\DeclareMathOperator{\covariance}{\mathrm{Cov}}
\DeclareMathOperator{\bias}{\mathrm{Bias}}


% TODO Prob TODO

% \renewcommand{\given}[1]{\mleft. \vphantom{##1} \;\bm{\mathord{\mathlarger{\middle\vert}}}\; \vphantom{##1} \mright.}
%\newcommand{\prob}[1]{\probability \bm{\mleft\lparen\vphantom{#1}\mright.} #1 \bm{\mleft.\vphantom{#1}\mright\rparen}}
\newcommand{\Prob}[1]{\probability \bm{\mathord{\mathlarger{\mleft\lparen\vphantom{#1}\mright.}}} #1 \bm{\mathord{\mathlarger{\mleft.\vphantom{#1}\mright\rparen}}}}
\newcommand{\Exp}[1]{\expectation \bracks*{#1}}
\newcommand{\Var}[1]{\variance \parens*{#1}}
\newcommand{\Cov}[1]{\covariance \parens*{#1}}
\newcommand{\Bias}[1]{\bias \parens*{#1}}

\newcommandx{\PDF}[3][1=f, 3=X]{{#1}_{#3} \parens*{#2}} % TODO
\newcommandx{\PMF}[3][1=p, 3=X]{{#1}_{#3} \parens*{#2}} % TODO
\newcommandx{\CDF}[3][1=F, 3=X]{{#1}_{#3} \parens*{#2}} % TODO


%
%  Indicator Variables
%
\DeclareMathOperator{\zero}{\mathds{0}}
\DeclareMathOperator{\one}{\mathds{1}}

\newcommandx{\indicator}[1][1=I]{\operatorname{\mathds{#1}}}


%
%  Discrete Univariate Distributions
%
\DeclareMathOperator*{\@macros@distribution@bernoulli@name}{\textrm{Bern}}
\DeclareMathOperator*{\@macros@distribution@binomial@name}{\textrm{Bin}}
\DeclareMathOperator*{\@macros@distribution@geometric@name}{\textrm{Geom}}
\DeclareMathOperator*{\@macros@distribution@negativebinomial@name}{\textrm{NBin}}
\DeclareMathOperator*{\@macros@distribution@hypergeometric@name}{\textrm{HGeom}}
\DeclareMathOperator*{\@macros@distribution@poisson@name}{\textrm{Pois}}
%\DeclareMathOperator*{\@macros@distribution@<>@name}{\textrm{}}

\newcommandx{\@macros@distribution@bernoulli@pmf}[1]{} % TODO
\newcommandx{\@macros@distribution@binomial@pmf}[1]{} % TODO
\newcommandx{\@macros@distribution@geometric@pmf}[1]{} % TODO
\newcommandx{\@macros@distribution@negativebinomial@pmf}[1]{} % TODO
\newcommandx{\@macros@distribution@hypergeometric@pmf}[1]{} % TODO
\newcommandx{\@macros@distribution@poisson@pmf}[1]{} % TODO
%\newcommandx{\@macros@distribution@<>@pmf}[1]{}

\newcommandx{\@macros@distribution@bernoulli@expectation}[1]{} % TODO
\newcommandx{\@macros@distribution@binomial@expectation}[1]{} % TODO
\newcommandx{\@macros@distribution@geometric@expectation}[1]{} % TODO
\newcommandx{\@macros@distribution@negativebinomial@expectation}[1]{} % TODO
\newcommandx{\@macros@distribution@hypergeometric@expectation}[1]{} % TODO
\newcommandx{\@macros@distribution@poisson@expectation}[1]{} % TODO
%\newcommandx{\@macros@distribution@<>@expectation}[1]{}

\newcommandx{\@macros@distribution@bernoulli@variance}[1]{} % TODO
\newcommandx{\@macros@distribution@binomial@variance}[1]{} % TODO
\newcommandx{\@macros@distribution@geometric@variance}[1]{} % TODO
\newcommandx{\@macros@distribution@negativebinomial@variance}[1]{} % TODO
\newcommandx{\@macros@distribution@hypergeometric@variance}[1]{} % TODO
\newcommandx{\@macros@distribution@poisson@variance}[1]{} % TODO
%\newcommandx{\@macros@distribution@<>@variance}[1]{}

\newcommandx{\@macros@distribution@bernoulli}[1]{} % TODO
\newcommandx{\@macros@distribution@binomial}[1]{} % TODO
\newcommandx{\@macros@distribution@geometric}[1]{} % TODO
\newcommandx{\@macros@distribution@negativebinomial}[1]{} % TODO
\newcommandx{\@macros@distribution@hypergeometric}[1]{} % TODO
\newcommandx{\@macros@distribution@poisson}[1]{} % TODO
%\newcommandx{@macros@distribution@<>}[1]{}


%
%  Discrete Multivariate Distributions
%
\DeclareMathOperator*{\@macros@distribution@categorical@name}{\textrm{Cat}}
%\DeclareMathOperator*{\@macros@distribution@<>@name}{\textrm{}}

\newcommandx{\@macros@distribution@categorical@pmf}[1]{} % TODO
%\newcommandx{\@macros@distribution@<>@pmf}[1]{}

\newcommandx{\@macros@distribution@categorical@expectation}[1]{} % TODO
%\newcommandx{\@macros@distribution@<>@expectation}[1]{}

\newcommandx{\@macros@distribution@categorical@variance}[1]{} % TODO
%\newcommandx{\@macros@distribution@<>@variance}[1]{}

\newcommandx{\@macros@distribution@categorical}[1]{} % TODO
%\newcommandx{@macros@distribution@<>}[1]{}


%
%  Continuous Univariate Distributions
%
\DeclareMathOperator*{\@macros@distribution@normal@name}{\textrm{N}}
\DeclareMathOperator*{\@macros@distribution@uniform@name}{\textrm{Unif}}
\DeclareMathOperator*{\@macros@distribution@exponential@name}{\textrm{Expo}}
\DeclareMathOperator*{\@macros@distribution@gamma@name}{\textrm{Gamma}}
\DeclareMathOperator*{\@macros@distribution@beta@name}{\textrm{Beta}}
\DeclareMathOperator*{\@macros@distribution@chisquare@name}{\chi^2}
\DeclareMathOperator*{\@macros@distribution@student@name}{\textrm{Stu}}
\DeclareMathOperator*{\@macros@distribution@weibull@name}{\textrm{Wbl}}
\DeclareMathOperator*{\@macros@distribution@multinomial@name}{\textrm{Mult}}
%\DeclareMathOperator*{\@macros@distribution@<>@name}{\textrm{}}

\newcommandx{\@macros@distribution@normal@cdf}[1]{} % TODO
\newcommandx{\@macros@distribution@uniform@cdf}[1]{} % TODO
\newcommandx{\@macros@distribution@exponential@cdf}[1]{} % TODO
\newcommandx{\@macros@distribution@gamma@cdf}[1]{} % TODO
\newcommandx{\@macros@distribution@beta@cdf}[1]{} % TODO
\newcommandx{\@macros@distribution@chisquare@cdf}[1]{} % TODO
\newcommandx{\@macros@distribution@student@cdf}[1]{} % TODO
\newcommandx{\@macros@distribution@weibull@cdf}[1]{} % TODO
\newcommandx{\@macros@distribution@multinomial@cdf}[1]{} % TODO
%\newcommandx{\@macros@distribution@<>@cdf}[1]{}

\newcommandx{\@macros@distribution@normal@expectation}[1]{} % TODO
\newcommandx{\@macros@distribution@uniform@expectation}[1]{} % TODO
\newcommandx{\@macros@distribution@exponential@expectation}[1]{} % TODO
\newcommandx{\@macros@distribution@gamma@expectation}[1]{} % TODO
\newcommandx{\@macros@distribution@beta@expectation}[1]{} % TODO
\newcommandx{\@macros@distribution@chisquare@expectation}[1]{} % TODO
\newcommandx{\@macros@distribution@student@expectation}[1]{} % TODO
\newcommandx{\@macros@distribution@weibull@expectation}[1]{} % TODO
\newcommandx{\@macros@distribution@multinomial@expectation}[1]{} % TODO
%\newcommandx{\@macros@distribution@<>@expectation}[1]{}

\newcommandx{\@macros@distribution@normal@variance}[1]{} % TODO
\newcommandx{\@macros@distribution@uniform@variance}[1]{} % TODO
\newcommandx{\@macros@distribution@exponential@variance}[1]{} % TODO
\newcommandx{\@macros@distribution@gamma@variance}[1]{} % TODO
\newcommandx{\@macros@distribution@beta@variance}[1]{} % TODO
\newcommandx{\@macros@distribution@chisquare@variance}[1]{} % TODO
\newcommandx{\@macros@distribution@student@variance}[1]{} % TODO
\newcommandx{\@macros@distribution@weibull@variance}[1]{} % TODO
\newcommandx{\@macros@distribution@multinomial@variance}[1]{} % TODO
%\newcommandx{\@macros@distribution@<>@variance}[1]{}

\newcommandx{\@macros@distribution@normal}[1]{} % TODO
\newcommandx{\@macros@distribution@uniform}[1]{} % TODO
\newcommandx{\@macros@distribution@exponential}[1]{} % TODO
\newcommandx{\@macros@distribution@gamma}[1]{} % TODO
\newcommandx{\@macros@distribution@beta}[1]{} % TODO
\newcommandx{\@macros@distribution@chisquare}[1]{} % TODO
\newcommandx{\@macros@distribution@student}[1]{} % TODO
\newcommandx{\@macros@distribution@weibull}[1]{} % TODO
\newcommandx{\@macros@distribution@multinomial}[1]{} % TODO
%\newcommandx{\@macros@distribution@<>}[1]{}


%
%  Continuous Multivariate Distributions
%
%\DeclareMathOperator*{@macros@distribution@<>@name}{\textrm{}}

%\newcommandx{\@macros@distribution@<>@cdf}[1]{}

%\newcommandx{\@macros@distribution@<>@expectation}[1]{}

%\newcommandx{\@macros@distribution@<>@variance}[1]{}

%\newcommandx{\@macros@distribution@<>}[1]{}


%
%  Statistical Estimation
%
\newcommand{\mean}{\bar}
\newcommand{\widemean}{\overline}
\newcommand{\median}{\tilde}
\newcommand{\widemedian}{\widetilde}
\newcommand{\sample}[1]{\hat{#1}}
\newcommand{\population}[1]{#1^{\ast}}
\newcommand{\param}[1]{\mathord{\mathsmaller{\mathrm{#1}}}}
\newcommand{\randvar}[1]{\mathord{\mathlarger{\mathsf{#1}}}}
\newcommand{\mle}{\mathrm{MLE}}
\newcommand{\map}{\mathrm{MAP}}


%
%  Optimization
%
\DeclareMathOperator*{\maximize}{\mathrm{maximize}}
\DeclareMathOperator*{\minimize}{\mathrm{minimize}}
\DeclareMathOperator*{\subjectto}{\mathrm{subject}\ \mathrm{to}}
\DeclareMathOperator*{\argmax}{\arg\max}
\DeclareMathOperator*{\argmin}{\arg\min}


%
%  Convex Optimization
%
\DeclareMathOperator*{\convex}{\mathrm{convex}}
\DeclareMathOperator*{\concave}{\mathrm{concave}}
%\DeclareMathOperator*{}{}


%%%%%%%%%%%%%%%%%%%%%%%%%%%%%%%%%%%%%%%%%%%%%%%%%%%%%%%%%%%%%%%%%%%%%%%%%%%%%%%
%%%%                                symbols                                %%%%
%%%%%%%%%%%%%%%%%%%%%%%%%%%%%%%%%%%%%%%%%%%%%%%%%%%%%%%%%%%%%%%%%%%%%%%%%%%%%%%


%
%  I dislike the default abbreviation of infinity...
%
\newcommand{\infinity}{\infty}
\newcommand{\posinf}{+\infty}
\newcommand{\neginf}{-\infty}

%
%  better qed symbol
%
\renewcommand{\qedsymbol}{$\blacksquare$}


%
%  ?
%
\newcommand{\cpm}{\mathbin{\mathpalette\@cpm\relax}}
\newcommand{\@cpm}[2]{\ooalign{%
  \raisebox{.1\height}{$#1+$}\cr
  \smash{\raisebox{-.6\height}{$#1-$}}\cr}}


% %
% %  matrix?
% %
% \renewcommand*\env@matrix[1][*\c@MaxMatrixCols\ c]{%
% \hskip -\arraycolsep\
% \let\@ifnextchar\new@ifnextchar\
% \array{#1}}


%%%%%%%%%%%%%%%%%%%%%%%%%%%%%%%%%%%%%%%%%%%%%%%%%%%%%%%%%%%%%%%%%%%%%%%%%%%%%%%
%%%%                                fonts                                  %%%%
%%%%%%%%%%%%%%%%%%%%%%%%%%%%%%%%%%%%%%%%%%%%%%%%%%%%%%%%%%%%%%%%%%%%%%%%%%%%%%%

%
%  importing v symbol from txfonts
%
\DeclareSymbolFont{matha}{OML}{txmi}{m}{it}% txfonts
\DeclareMathSymbol{\varv}{\mathord}{matha}{118}


%
%  importing double stroke \perp symbol from txfonts
%
\DeclareSymbolFont{symbolsC}{U}{txsyc}{m}{n}
\SetSymbolFont{symbolsC}{bold}{U}{txsyc}{bx}{n}
\DeclareFontSubstitution{U}{txsyc}{m}{n}
\DeclareMathSymbol{\dperp}{\mathrel}{symbolsC}{121}


%
%  lowercase mathcal font
%
\newcommand{\smallmathcal}[1]{%
  \mathchoice
    {{\scriptstyle\mathcal{\uppercase{#1}}}} % \displaystyle
    {{\scriptstyle\mathcal{\uppercase{#1}}}} % \textstyle
    {{\scriptscriptstyle\mathcal{\uppercase{#1}}}} % \scriptstyle
    {\scalebox{.7}{\(\scriptscriptstyle\mathcal{\uppercase{#1}}\)}} % \scriptscriptstyle
}


%
%  math font abbreviations
%
\newcommand{\mbf}[1]{\mathbf{#1}}
\newcommand{\mbb}[1]{\mathbb{#1}}
\newcommand{\mcal}[1]{\ifthenelse{\isin{#1}{AÂBCÇDEFGĞHIİÎJKLMNOÖÔPRSŞTUÜÛVYZ}}{\mathcal{#1}}{\smallmathcal{#1}}}


%%%%%%%%%%%%%%%%%%%%%%%%%%%%%%%%%%%%%%%%%%%%%%%%%%%%%%%%%%%%%%%%%%%%%%%%%%%%%%%
%%%%                               functions                               %%%%
%%%%%%%%%%%%%%%%%%%%%%%%%%%%%%%%%%%%%%%%%%%%%%%%%%%%%%%%%%%%%%%%%%%%%%%%%%%%%%%

%
%  switch statement implementation
%
\newcommand{\ifequals}[3]{\ifthenelse{\equal{#1}{#2}}{#3}{}}
\newcommand{\caseof}[2]{#1 #2}
\newenvironment{switch}[1]{\renewcommand{\caseof}{\ifequals{#1}}}{}


%
% sequence generator [SEP] {LIST} [FUNC]
%
\newcommandx*{\mkseq}[3][1=\, , 3=\idfunc]{%
	\foreach \elem [count=\i] in {#2} {%
		\ifnum\i=1%
			#3{\elem}%
		\else%
			#1 #3{\elem}%
		\fi%
	}%
}%


% %
% % range generator [START] {VAR} [STOP] [SKIPABOVE]
% %       usage:
% %          -  \mkseq[SKIPABOVE]{\mkrange[START]{VAR}[STOP]}[SEP][FUNC]
% %
% \newcommandx*{\@preamble@mkrange@body}[4][]{%
%   \foreach \elem [count=\i] in {#1,...,\@preamble@mkrange@above} {%
%     \ifnum\i = 1%
%       #2_{\elem}%
%     \else%
%       \ifnum\i = #4%
%         \ifthenelse{\boolean{@preamble@mkrange@tail}}{%
%           \, #2_{\elem) \, \ldots \, #2_{#3}%
%         }{%
%           \, #2_{\elem}%
%         }%
%       \else%
%         \, #2_{\elem}%
%       \fi%
%     \fi%
%   }%
% }%
% \newcommandx*{\mkrange}[4][1=1 , 3=n, 4=2]{%
%   \IfInteger{#1}{% START = INT
%     \IfInteger{#4}{% SKIPABOVE = INT
%       \def\@preamble@mkrange@above{\xintexpr #1 + #4\relax}%
%       \newboolean{@preamble@mkrange@tail}
%       \setboolean{@preamble@mkrange@tail}{true}
%       \IfInteger{#3}{% STOP = INT
%         \ifthenelse{#1 > #3}{% START > STOP
%           \PackageError{preamble}{mkrange.START = '#1' > mkrange.STOP = '#3'}\@ehc%
%         }{%
%           \ifthenelse{#1 = #3}{% START = STOP
%             #2_{#1}%
%           }{% START < STOP
%             \ifthenelse{#4 = #3}{% SKIPABOVE = STOP
%               \boolean{@preamble@mkrange@tail}{false}
%               \@preamble@mkrange@body{#1}{#2}{#4}{#3}%
%             }{%
%               \ifthenelse{#4 < #3}{% SKIPABOVE < STOP
%                 \@preamble@mkrange@body{#1}{#2}{#4}{#3}%
%               }{% SKIPABOVE > STOP
%                 \PackageError{preamble}{mkrange.SKIPABOVE = '#4' < mkrange.STOP = '#3'}\@ehc%
%               }%
%             }%
%           }%
%         }%
%       }{% STOP != INT
%         \ifthenelse{#4 < #1}{% START > SKIPABOVE
%           \PackageError{preamble}{mkrange.SKIPABOVE = '#4' < mkrange.START = '#1'}\@ehc%
%         }{% START <= SKIPABOVE,  STOP != INT
%           \@preamble@mkrange@body{#1}{#2}{#4}{#3}
%         }%
%       }%
%     }{\PackageError{preamble}{mkrange.SKIPABOVE = '#4' is not an Integer}\@ehc}%
%   }{\PackageError{preamble}{mkrange.START = '#1' is not an Integer}\@ehc}%
% }%


%%%%%%%%%%%%%%%%%%%%%%%%%%%%%%%%%%%%%%%%%%%%%%%%%%%%%%%%%%%%%%%%%%%%%%%%%%%%%%%
%%%%                              environments                             %%%%
%%%%%%%%%%%%%%%%%%%%%%%%%%%%%%%%%%%%%%%%%%%%%%%%%%%%%%%%%%%%%%%%%%%%%%%%%%%%%%%

%
%  cases
%
\newenvironment{case}[1]{\left\lbrace\begin{array}{lr}#1}{\end{array}\right\rbrace}


% %
% % Create boxed Environment
% %
% \newenvironment{boxed}%
% 	{%
% 		\begin{adjustbox}{minipage=0.85\textwidth,precode=\dbox}
% 	}{%
% 		\end{adjustbox}
% 	}%


%
% Create boxedSolution Environment
%
\newenvironment{boxedSolution}%
  {%
    \\
    \begin{solution}
    \\
    \begin{adjustbox}{minipage=.85\textwidth,precode=\dbox}
  }{%
    \end{adjustbox}
    \end{solution}
  }%


%
% Create boxedProof Environment
%
\newenvironment{boxedProof}%
  {%
    \begin{adjustbox}{minipage=.85\textwidth,precode=\dbox}
    \begin{proof}
  }{%
    \end{proof}
    \end{adjustbox}
  }%


%%%%%%%%%%%%%%%%%%%%%%%%%%%%%%%%%%%%%%%%%%%%%%%%%%%%%%%%%%%%%%%%%%%%%%%%%%%%%%%
%%%%                             SML Pseudocode                            %%%%
%%%%%%%%%%%%%%%%%%%%%%%%%%%%%%%%%%%%%%%%%%%%%%%%%%%%%%%%%%%%%%%%%%%%%%%%%%%%%%%

\newdimen\zzsize
\zzsize=11pt
\newdimen\kwsize
\kwsize=11pt

\newcommand{\basicstyle}{\fontsize{\zzsize}{1.1\zzsize}\ttfamily}
\newcommand{\keywordstyle}{\fontsize{\kwsize}{1.1\kwsize}\normalfont\bf}
\newlength{\zzlstwidth}
\settowidth{\zzlstwidth}{{\basicstyle~}}
\newcommand{\lcm}{}

\lstset,
  lineskip={1.5pt},
  columns=[l]fullflexible,
  keepspaces=true,
  mathescape=true,
  escapeinside={@}{@},
% NOTE: need TWO sets of braces around each definition below!
  literate={requires}{{$\lcm\text{\keywordstyle \% requires}$}}6
           {returns}{{$\lcm\text{\keywordstyle \% returns}$}}6
           {=}{{$\lcm=$}}2
           {(}{{$($}}2
           {)}{{$)$}}2
           {**}{{$\lcm\times$}}2
%           {||}{{$\lcm\Vert$}}2
           {|}{{$|$}}2
           {=>}{{$\lcm\boldsymbol\Rightarrow$}}2
           {->}{{$\lcm\rightarrow$}}2
           {'a}{{$\alpha$}}1
           {'b}{{$\beta$}}1
}


%%%%%%%%%%%%%%%%%%%%%%%%%%%%%%%%%%%%%%%%%%%%%%%%%%%%%%%%%%%%%%%%%%%%%%%%%%%%%%%
%%%%%%%%%%%%%%%%%%%%%%%%%%%%%%%%%%%%%%%%%%%%%%%%%%%%%%%%%%%%%%%%%%%%%%%%%%%%%%%
